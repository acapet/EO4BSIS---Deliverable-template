%%%%%%%%%%%%%%%%%%%%%%%%%%%%%% -*- Mode: Latex -*- %%%%%%%%%%%%%%%%%%%%%%%%%%%%
%%% executive-summary.tex --- 
%% 
%% Filename: executive-summary.tex
%% Description: 
%% Authors: Mohamed Ahmed <m.ahmed@cs.ucl.ac.uk> @CHANGE / Emil Slusanschi <emil.slusanschi@cs.pub.ro> @UNICORE
%% Modification A. Capet, ULiège, for the EO4BSIS ESA project
%% 
%%%%%%%%%%%%%%%%%%%%%%%%%%%%%%%%%%%%%%%%%%%%%%%%%%%%%%%%%%%%%%%%%%%%%%
%% 
%%% Change log:
%% $Date$
%% $Revision$
%% $Id$
%% 
%%%%%%%%%%%%%%%%%%%%%%%%%%%%%%%%%%%%%%%%%%%%%%%%%%%%%%%%%%%%%%%%%%%%%%
\chapter*{Executive Summary}
\addcontentsline{toc}{chapter}{Executive Summary}

Please include an executive summary (maximum 2 pages).

\textbf{Hints for writing an executive Summary:} 
Summaries are useful for people who have neither the time nor the inclination to
read a lengthy document but who want to scan the primary points quickly and then
decide whether they need to read the entire version. Because they are often
geared to busy managers, we call them executive summaries.

A summary should be short enough to be economical and long enough to be clear and
comprehensive. Don't sacrifice meaning for brevity. A short, confusing summary
will take more of a busy executive's time than a somewhat longer but clear one.

\textbf{Capture the essential meaning of the original document}  
A good summary will always tell the reader what the original says-its significant
points, primary findings, important names, numbers, and measurements, and major
conclusions and recommendations. The essential message is the minimum that the
reader needs to understand the shortened version of the whole. The essential
meaning does not include background information, lengthy examples, visuals, or
long definitions.

\textbf{Write at the lowest level of specialisation}  
If the executive summary is part of a report, more people may read the summary
than the entire report. Write at the lowest level of technicality, translating
specialized terms and complex data in to plain English because your summary will
not include the supporting information for technical statements. If you know your
audience, keep these people in mind. When in doubt, oversimplify.

\textbf{Structure the summary to fit your audience's requirements} 
Some summaries follow the organisation of the report, dealing briefly with the
information in each chapter (or section) in order. Others highlight the findings,
conclusions, and recommendations by summarising them first, before going on to
discuss procedures or methodologies. If you are writing a summary at the request
of your manager, you may want to begin with the part that seemed to be of most
interest to him or her.

\textbf{Avoid introducing new data into the summary}  
Represent the original faithfully. An executive summary is not a book
report. Avoid personal comments such as ``This report was very interesting'', or
``The author seems to think that$\ldots$'' You don't need to try to put the work
into a particular perspective.

\textbf{Write your summary so that it can stand alone} 
Your summary should be a self-contained message. Your reader should read the
original only if he or she wants to get a fleshed-out view of the subject-not to
make sense out of what you have said in your summary.

\begin{itemize}
\item Read the entire original before writing a word. Get the complete picture. 
\item Re-read and underline significant points (usually in the topic sentence of each paragraph). 
\item Re-write in your own words, listing all significant points. 
\item Edit your draft, cutting needless words and phrases.
\end{itemize}

%%%%%%%%%%%%%%%%%%%%%%%%%%%%%%%%%%%%%%%%%%%%%%%%%%%%%%%%%%%%%%%%%%%%%% 
%%% executive-summary.tex ends here

%%% Local Variables: 
%%% mode: latex
%%% TeX-master: t
%%% End: 
